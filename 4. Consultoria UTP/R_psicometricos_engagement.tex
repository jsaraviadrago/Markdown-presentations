\documentclass{beamer}
\usetheme{Boadilla}

\title{Resultados psicométricos prueba engagement}
\author{Juan Carlos Saravia}
\date{}



\begin{document}

\begin{frame}
\titlepage
\end{frame}



\begin{frame}
\frametitle{Objetivos}
Evaluar las propiedades psicométricas para la prueba de Engagement de los estudiantes de UTP
\end{frame}



\begin{frame}
\frametitle{Descripción de la muestra}
\begin{itemize}
	\item N = 4309 
	\item Casos únicos: 2274 
	\item Estudiantes de 6 cursos distintos de primer año del turno noche
	\item Los estudiantes llevan cursos virtuales 
\end{itemize}
\end{frame}	

\begin{frame}
\frametitle{Descripción de la muestra}
Inglés, Química introducción a la vida universitaria tienen algunas características importantes que podrían influir en los resultados: \\

\begin{itemize}
	\item Inglés es asíncronico y siempre virtual
	\item Química es un curso mixto donde tiene presencial y virtual
	\item Introducción a la vida universitario es un curso más accesible donde no hay una tendencia a desaprobar a los estudiantes
\end{itemize}	
\end{frame}

\begin{frame}
\frametitle{Descripción de la prueba}
\begin{itemize}
	\item La versión de la prueba que se aplicó está compuesta por 20 ítems y todos los reactivos positivos. 
	\item La prueba consta de 4 áreas: Conductual, Emocional, Agencial y Cognitivo
	\item La prueba no tiene un puntaje único
\end{itemize}	


\alert{\textbf{Nota importante:}} se retiraron los casos duplicados para los análisis psicométrico y se tomaron los estudiantes con las respuestas más completas. 

\end{frame}

\begin{frame}
\frametitle{Propuesta analítica}

\begin{itemize}
\item Normaliad Multivariada: test de Mardia\footnote{El test de Mardia es un clásico test para revisar la normalidad multivariada que es una condición importante para hacer análisis factoriales confirmatorios}	
\item Análisis factorial confirmatorio: se utilizó el algoritmo de WLSMV\footnote{Este algoritmo solo se utiliza cuando no hay normalidad multivariada o las pruebas son ordinales como es el caso de las escalas Likert}	
\item Se tomaron los criterios de Hu y Bentler para ver cuán bien fue el ajuste general de los ítems en la prueba
\item Se tomó el criterio de Kline de tener una relación mayor a 0.40 entre los ítems y el puntaje total de la escala


\end{itemize}	

\end{frame}

\begin{frame}
\frametitle{Resultados}


\begin{block}{No se encontró normalidad multivariada}
\begin{itemize}
\item Esto implica que si se juntan todos los ítems en conjunto no se forma una distibución normal en forma de campana
\end{itemize}
\end{block}

\begin{examples}{Análisis factorial confirmatorio}
\begin{itemize}
\item Todos los ítems ajustaron correctamente y tienen una relación adecuada con el puntaje total de la prueba
\item Todos los indicadores de bondad de ajuste mostraron resultados adecuados 
\end{itemize}
\end{examples}

\end{frame}

\end{document}