\documentclass{beamer}
\usetheme{Boadilla}

\title{Analisis psicométricos prueba engagement}
\author{Juan Carlos Saravia}



\begin{document}

\begin{frame}
\titlepage
\end{frame}


%Objetivos especificos 1
\section{Evaluar las propiedades psicométricas de la prueba de engagement}
\subsection{Evaluar las evidencias de la estructura interna de los ítems}
\subsection{Proponer el mejor modelo de la prueba de engagement para los datos de UTP}

%Objetivos especificos 2
\section{Describir los resultados de engagement}
\subsection{Por estratos}
\subsection{Por curso}
\subsection{Revisar la variabilidad del engagement de los estudiantes en diferentes cursos}


\begin{frame}
\frametitle{Objetivos}
Evaluar las propiedades psicométricas para la prueba de Engagement y revisar las respuestas de la de los estudiantes según estratos de UTP
\end{frame}



\begin{frame}
\frametitle{Objetivos especificos}
\tableofcontents
\end{frame}

\begin{frame}
\frametitle{Propuesta analítica}

\begin{itemize}
\item Análisis factorial confirmatorio	
\item Análisis descriptivo: medidas de tendencia central y dispersión. 
\end{itemize}	

\end{frame}

\begin{frame}
\frametitle{Entregables}

Los \alert{entregables} para este producto serán los siguientes 


\begin{block}{Solución psicométrica de la escala de satisfacción}
\begin{itemize}
\item Código con el análisis realizado
\item Resultados de los análisis psicométricos
\end{itemize}
\end{block}

\begin{block}{Descriptivos de la escala de satisfacción}
\begin{itemize}
\item Resultados por estratos de escala de satisfacción
\item Variabilidad entre el engagement por curso
\end{itemize}
\end{block}




\end{frame}


\end{document}