\documentclass{beamer}
\usetheme{Boadilla}

\title{Ficha técnica de descriptivos de prueba de engament}
\author{Juan Carlos Saravia}
\date{}



\begin{document}

\begin{frame}
\titlepage
\end{frame}

%Objetivos especificos 1
\section{Describir con promedio las cuatro áreas de engagement}
\subsection{Describir por curso}
\subsection{Describir por curso y estrato}

%Objetivos especificos 2
\section{Describir la relación de engagement de los esstudiante que respondieron para cursos de ciencias y humanidades}
\subsection{Calcular los puntajes de las cuatro áreas de engagement para ciencias y humanidades}
\subsection{Describir la relación entre engagmeent de ciencias y humanidades en las cuatro áreas de la prueba}



\begin{frame}
\frametitle{Objetivos}
Describir los resultados de la prueba de engagement 
\end{frame}

\begin{frame}
\frametitle{Objetivos especificos}
\tableofcontents
\end{frame}



\begin{frame}
\frametitle{Descripción del proceso}
\begin{itemize}
	\item Se calculó el promedio de engagment Conductual, Emocional, Agencial y Cognitivo. 
	\item El puntaje se calculó sumando todos los ítems y diviendolo entre la cantidad. Los valores van del 1-7 así como los ítems
	\item El promedio se realizo por curso y estrato en ese caso no hubo respuesta duplicadas de los estudiante por curso. 
	\item Los estudiantes llevan cursos virtuales 
\end{itemize}
\end{frame}	


\end{document}