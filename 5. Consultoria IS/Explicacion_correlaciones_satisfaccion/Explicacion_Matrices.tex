\documentclass{beamer}
\usetheme{Boadilla}

\title{Explicacion matrices satisfaccion general con características}
\author{Juan Carlos Saravia}
\date{}



\begin{document}

\begin{frame}
\titlepage
\end{frame}


%Objetivos especificos 1
\subsection{Realizar los análisis descriptivos del modelo d. e deserción}
\subsection{Construir el modelo de deserción de estudiantes}


\begin{frame}
\frametitle{Objetivos}
Describir la interpretación de las correlaciones por elemento
\end{frame}



\begin{frame}
\frametitle{Descripcion del análisis}
El análiis de correlación es una técnica que busca revisar la relación en línea recta que hay entre dos variables. En este caso, dicho análisis se visualiza con un gráfico llamada diagrama de dispersión. Como su nombre lo dice muestra cuán dispersos están cada una de las escuelas en dos ejes, el vertical y el horizontal. 

En ese caso, cuando se ve un gráfico se puede encontrar los resultados de cada escuela en forma de puntos por variable y una línea de tendencia que es en promedio cuánto se relaciona una varible u otra. 

Adicionalmente se agregó dos líbeas (vertical y horizontal) que marcan el promedio de ambas características. Con ello se pueden formar cuatro cuadrantes. 
\end{frame}

\begin{frame}
\frametitle{Pasos para leer el gráfico}

\begin{itemize}
\item Detectar la escuela que uno está interesado y revisar si es que está por encima o por debajo de la línea de tendencia. 
\item Identificar el cuadrante en el que está la escuela de interés 
\end{itemize}	

\end{frame}

\begin{frame}
\frametitle{¿Cómo se lee e interpreta?}


\begin{itemize}
\item Si un punto o escuela está por encima de la línea de tendencia es que sus puntajes están por encima de lo esperado, en relación a ambas variables. Sin embargo, acá surge la pregunta ¿cuán por encima? Para ello hay que mirar lo cuadrantes.
\item Si una escuela está en el cuadrante superior o inferior derecho quiere decir que los puntajes de esa escuela están por encima del promedio de la variable del eje horizontal (por ejemplo si es satisfacción con la metodología, los puntajes están por encima del promedio de esa variable). Si son los cuadrantes de la izquierda tanto el inferior como el superior los puntajes en este caso de satisfacción con la metodología están por debajo del promedio de todas las escuelas. 

\end{itemize}
\end{frame}

\begin{frame}
\frametitle{¿Cómo se lee e interpreta?}
\begin{itemize}
\item Ahora, ¿qué pasa con los cuadrantes superiores e inferiores? Los cuadrantes superior dicen que esas escuelas tienen puntajes por encima del promedio en satisfacción general. Los inferiores tienen puntajes por debajo del promedio.
\item La escuela ideal así, es la escuela que está en el cuadrante superior derecho y que además está encima de la línea de tendencia. 
\item La escuela problemática es la que está en el cuadrante inferior izquierdo pero que además está por debajo de la línea de tendencia diagonal. 

\end{itemize}

\end{frame}




\end{document}